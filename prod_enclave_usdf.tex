 
 \subsection{USDF}

\subsubsection{US Data Facility as Hybrid Cloud-on prem}

 The USDF is envisaged as being a hybrid model: science users will
interact with the RSP, hosted in the cloud, while processing and
storage archive will be on-prem at SLAC. The cloud portion  will be a
continuation of the successful IDF demonstrated in DP0 and act as the
US DAC.

\begin{figure}
\begin{centering}
\includegraphics[width=0.9\textwidth]{images/HybridUSDF.jpg}
	\caption{USDF as hybrid Cloud-on prem\label{fig:HybridUSDF}}
\end{centering}
\end{figure}

Needed services will be to populate the cloud storage with a data
cache from the on-prem archive, and to provide additional user batch
resources outside of the cloud resources.
 
 
\subsubsection{Replacing NCSA as US Data Facility}

NCSA shut down for Rubin at the end of August 2022. In order to take
over, the USDF:
\begin{itemize}
\item Transferred some 5 PB of historical data (raw data, processing
output and user data) to SLAC
\item Onboarded some 200 staff and in-kind commissioners, providing a
similar environment to NCSA:
\begin{itemize}
\item developer nodes
\item batch system with 2000 cores (also usable via HTCondor and Parsl)
\item instance of the Rubin Science Platform
\item Butler instance
\item at the time of transfer, workflow (PanDA) was still using the
CERN instance used by the IDF for DP0.2, and a small Rucio instance
for testing multi-site transfers.
\item live sync'ed EFD
\item data transfers from the summit to SLAC (eg for AuxTel and ComCam)
\end{itemize}
\end{itemize}

The USDF is hosted by the SLAC Shared Science Data Facility (S3DF). It
provides:
\begin{itemize}
\item slurm batch system
\item bastion host login nodes (to jump to Rubin developer nodes)
\item weka file system, which provides a tiered flash/spindle storage
system with POSIX and S3 direct access.
\item tape silo using HPSS for backup
\item data transfer nodes (DTNs)
\end{itemize}

User home directories, group space and software are stored on flash;
each SLAC project provides its own storage for data in the weka tiered system.

\subsubsection{Supporting Commissioning}

During commissioning, the data from the EFD, rapid analysis ran on site, and the single-frame measurement pipeline ran at the USDF (as discussed in section \ref{sec:commissioning_scope}) needs to be rendezvoused into a common database, at least as viewed by the user. 
This includes the ability to have data sorted per visit, average time, or for a given calibration set.
The requirements for this database (or databases) are currently being refined.

Also undergoing definition, but will require support, is for ``Processing and Analysis Management."
This is where datasets are selected and processed based on data quality and/or other criteria.
Identifiers are then attached (e.g. quality flags) to dataIds which indicate units of data (e.g. (visit, ccd)) which can be used in future processing.



- Need consolidated database
- Process and analysis management 
Backed by a database



\subsubsection{Supporting DRP Workflow and Data Movement}

Production servers will be instanciated at SLAC for PanDA and Rucio,
both deployed on kubernetes, both backed by postgres databases.

\subsubsection{Resources Projection}

The USDF follows the DMTN-135 sizing model for survey operation,
essentially adding 2000-2500 compute cores and 30 PB each of disk and
tape for each survey year. Buildup for survey is expected to start in FY24.

Resources for ComCam and LSSTCam support in commissioning will provide
about 5000 cores and 15 PB of storage total.

\subsubsection{Alert and Prompt Processing, and Alerts Distribution}

A prompt processing harness is being developed on the IDF and expected to be
available for port to the USDF in the last quarter of 2022. This will
provide the kafka feed mechanism that brokers will be connecting to.

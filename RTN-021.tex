\documentclass[DM,authoryear,toc]{lsstdoc}
% lsstdoc documentation: https://lsst-texmf.lsst.io/lsstdoc.html
\input{meta}

% Package imports go here.

% Local commands go here.

%If you want glossaries
%\input{aglossary.tex}
%\makeglossaries

\title{USDF transition plan}

% Optional subtitle
% \setDocSubtitle{A subtitle}

\author{%
William O'Mullane
}

\setDocRef{RTN-021}
\setDocUpstreamLocation{\url{https://github.com/lsst/rtn-021}}

\date{\vcsDate}

% Optional: name of the document's curator
% \setDocCurator{The Curator of this Document}

\setDocAbstract{%
This document outlines the plan to get the USDF fucntional for operations. 
}

% Change history defined here.
% Order: oldest first.
% Fields: VERSION, DATE, DESCRIPTION, OWNER NAME.
% See LPM-51 for version number policy.
\setDocChangeRecord{%
  \addtohist{1}{YYYY-MM-DD}{Unreleased.}{William O'Mullane}
}


\begin{document}

% Create the title page.
\maketitle
% Frequently for a technote we do not want a title page  uncomment this to remove the title page and changelog.
% use \mkshorttitle to remove the extra pages

% ADD CONTENT HERE
% You can also use the \input command to include several content files.

\appendix
% Include all the relevant bib files.
% https://lsst-texmf.lsst.io/lsstdoc.html#bibliographies
\section{References} \label{sec:bib}
\renewcommand{\refname}{} % Suppress default Bibliography section
\bibliography{local,lsst,lsst-dm,refs_ads,refs,books}

% Make sure lsst-texmf/bin/generateAcronyms.py is in your path
\section{Acronyms} \label{sec:acronyms}
\addtocounter{table}{-1}
\begin{longtable}{p{0.145\textwidth}p{0.8\textwidth}}\hline
\textbf{Acronym} & \textbf{Description}  \\\hline

AP & Alert Production \\\hline
APDB & Alert Production DataBase \\\hline
CE & Communications Engagement \\\hline
CPU & Central Processing Unit \\\hline
ComCam & The commissioning camera is a single-raft, 9-CCD camera that will be installed in LSST during commissioning, before the final camera is ready. \\\hline
DC2 & Data Challenge 2 (DESC) \\\hline
DESC & Dark Energy Science Collaboration \\\hline
DF & Data Facility \\\hline
DM & Data Management \\\hline
DP & Data Production \\\hline
DP0 & Data Preview 0 \\\hline
DP1 & Data Preview 1 \\\hline
DP2 & Data Preview 2 \\\hline
DR1 & Data Release 1 \\\hline
DRP & Data Release Production \\\hline
EFD & Engineering and Facility Database \\\hline
EPO & Education and Public Outreach \\\hline
FY21 & Financial Year 21 \\\hline
FrDF & French Data Facility \\\hline
IDF & Interim Data Facility \\\hline
IN2P3 & Institut National de Physique Nucléaire et de Physique des Particules \\\hline
L2 & Lens 2 \\\hline
L3 & Lens 3 \\\hline
LSST & Legacy Survey of Space and Time (formerly Large Synoptic Survey Telescope) \\\hline
MPC & Minor Planet Center \\\hline
NCSA & National Center for Supercomputing Applications \\\hline
PPDB & Prompt Products DataBase \\\hline
PR & Pull Request \\\hline
PanDA &  Production ANd Distributed Analysis system \\\hline
QA & Quality Assurance \\\hline
RDO & Rubin Directors Office \\\hline
RSP & Rubin Science Platform \\\hline
RTN & Rubin Technical Note \\\hline
SC & Science Collaboration \\\hline
SLAC & SLAC National Accelerator Laboratory \\\hline
SP & Story Point \\\hline
UKDF & United Kingdom Data Facility \\\hline
USDF & United States Data Facility \\\hline
\end{longtable}

% If you want glossary uncomment below -- comment out the two lines above
%\printglossaries





\end{document}

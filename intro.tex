\section{Introduction}

This document describes the preparations that are required to be ready to process telescope data into science-ready products in Operations and in the run-up to operations (e.g., during Commissioning). This is commonly referred to as Data Release Processing.

The document covers a period of transition, for the telescope, from Construction into Operations. While on the high-level Project roadmap the move from Construction to Operations is reasonably discrete, in practice it is a relatively complicated with a need to seamlessly maintain operational capabilities that have been implemented during Construction -- such as, for the processing of telescope data.

Data Release Processing is intended to be completed across three independent Data Facilities:

\begin{itemize}

\item The United States Data Facility (USDF), hosted at the SLAC National Accelerator Laboratory in California.
  
\item The French Data Facility, at the Institut national de physique nucléaire et de physique des particules (IN2P3) in Lyon.

\item The UK Data Facility hosted by the IRIS distributed infrastructure.

\end{itemize}

These Data Facilities are in different states of preparation, and also have different responsibilities during -- and in the run-up to -- Operations.

In the next section, we explain in high-level terms the key elements of DRP and the strategy that has been adopted for addressing them.

Then, in Section ???, we describe the plan for identifying, testing, and incorporating the technologies that are needed to support a three-facility DRP, and define the timeline on which this plan needs to be completed.

In Section ???, we consider the preparations that are in place at each, individual Data Facility --- typically within the control of local staff --- which are required to ensure each Facility achieves various stages of technology and infrastructure readiness, as the Observatory scales up to full operations.


% USDF operations are covered in the operations plan.
% However we need \gls{USDF} in place ahead of Rubin  operations so it
% can be fully functional on day one.
% This will require some initial setup and running some services in parallel with \gls{NCSA}.
% Some tests  run at NCSA will have to be rerun to verify requirements at the \gls{USDF}.

% This document captures the time line, structure and plan for getting the Data Facilities implemented.

% Success in the setup up of the \gls{USDF} is a shared responsibility between SLAC and Rubin Observatory,
% It is a preops activity but it has influence on some remaining
% construction tasks (see \secref{sec:construct}). Similarly, the annual
% Data Release Processings are a joint function of the three Data
% Facilities (France, UK, US).

% \subsection{Enclaves and functionality}\label{sec:enclaves}

% A complete service list  and how the services relate to enclaves is provided in
% \appref{sec:services}. In brief, these enclaves include: Prompt US
% Enclave; Offline Production Enclave; Archive US Enclave; US Data
% Access Center; and Development and Integration Enclave. Only the
% Offline Production enclave is required for the French and UK facilities.

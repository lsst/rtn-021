\section{Planning}\label{sec:plan}

The bottom layer of the Data Facilities is the hardware on which the
platforms run. These require a scalable architecture with sufficient
storage and \gls{CPU} to support the Data Preview/Release timeline.  This
element requires a design for the architecture followed by an
acquisition and installation plan.

The middle layer includes the infrastructure to support deployment of
hardware and tools for data movement and workflow management.

 The top layer involves the applications: science platform, \gls{Qserv} and pipelines.

In response to the timeline \secref{sec:timeline} the plan is as follows.

\subsection {Hardware architecture and technology}
See also \citeds{DMTN-189} (scope) and \citeds{DMTN-135}  (sizing).

\subsection{Key initial services}

These initial services/resources are planned to support \gls{DP1}. Support
for \gls{DP2} and DR1 are largely by increments in hardware per the sizing model.

\begin{itemize}
  \item Hardware
  \begin{itemize}
    \item file systems: An architecture choice must be made for object
      store, likely between ceph and minIO. This may affect the
      hardware choice (\gls{JBOD} vs appliance). 3.5 PB of object store and
      1.5 \gls{PB} of POSIX disk are envisaged.
    \item \gls{CPU} allocation: The bulk of the \gls{CPU} is in batch (1000 cores)
      and staff \gls{RSP} instances (500 cores).
      \item \gls{Qserv}: depending on the ultimate location of \gls{Qserv}, we
        expect to do scale testing in the cloud and at \gls{NCSA} prior to
        deciding on an implementation.
  \end{itemize}

\item
  \item \gls{Science Platform}
  \begin{itemize}
      \item \gls{K8S} is the standard for deploying applications and
        resources. The \gls{Science Platform} is built on top of
        it. Additionally, \gls{CI} activities are run via \gls{K8S}.
      \item \gls{RSP} has been installed in multiple locations and
        architectures. For \gls{DP1}, we expect science users to go to the
        IDF for data access, while the \gls{USDF} provides staff access.
  \end{itemize}

\item Workflow and Data Movement Tools
\begin{itemize}
      \item \gls{PanDA} is under serious consideration as the toolkit
        for at-scale workflow. It will get its first load testing in
        \gls{DP0}.2. It is expected that there will be worked needed to
        customize \gls{PanDA} to Rubin's situation. We also anticipate a
        layer on top of \gls{PanDA} to orchestrate campaign management.
	\item \gls{Rucio} is under consideration for data movement. It
       works with policies to schedule data movement and integrates
       with a transport layer (most commonly \gls{FTS}).
  \end{itemize}

\end{itemize}

\subsection {Enclave deployment}

The USDF depends on an expansion of SLAC's SRCF-I data center
(``SRCF-II''), which is scheduled for completion in March 2023, with
an estimated 6 months needed to be ready for hardware installations.

\subsubsection{Prompt}

DP1 drives much of the USDF implementation. A difference from DP2 is
that DP1 will feature only canned alerts. 

\paragraph{ DP1}

 \begin{itemize}
  \item Prompt processing requires a cluster of compute nodes, of
    relatively fixed size.
  \item Kubernetes
  \item \gls{APDB} - Cassandra database
  \item butler repository
  \item Kafka database for alert distribution
  \item transfer mechanism for summit images to USDF
  \item PanDA server
  \item Data BackBone services
 \end{itemize}

 \paragraph{DP2}

 \begin{itemize}
 \item connection to Minor Planet Center
 \item prompt processing cluster - 1200 cores
  \end{itemize}

 \paragraph{DR1}

 \begin{itemize}
 \item increase cluster and storage sizes appropriate to DR1 sizing
 \end{itemize}

\subsubsection{Archive}

\paragraph{DP1}

\begin{itemize}
\item Data BackBone services
\item Prompt Products database (Postgres)
 \item sufficient storage (1500 cores; 1.5 PB POSIX, 3.5 PB object store)
 \end{itemize}

 \paragraph{DR1}
 

 \begin{itemize}
 \item increased storage (2000 cores; 60 PB disk and tape)
 \end{itemize}
 
\subsubsection{US Data Access}

The DAC relies almost entirely on the \gls{RSP}, which draws data from
Qserv and the butler. Any A\&A issues will need to have been addressed
for the target users. These are all needed to be in place for DP1.

There may be distinct RSPs for staff and science users.

\subsubsection{Developer and Integration}

This enclave requires a staff RSP coupled to sufficient batch and
storage resources. The install required for DP1 should satisfy these needs.

\subsubsection{Offline Production}

\paragraph{USDF}


All the services needed for Offline Production have been described
above as needed in other enclaves: here, the USDF needs to add to its
hardware base to satisfy each phase.

\bf{DP1}

\begin{itemize}
 \item sufficient cores and storage (using 1000 cores; 3.5 PB object store)
 \end{itemize}

 \subsection{FrDF}

\subsubsection{Overview}

The Rubin French Data Facility (FrDF) is hosted and operated by IN2P3 / CNRS
computing center (\href{https://cc.in2p3.fr/en/}{CC-IN2P3}),
located in Lyon, France. This is a scientific data processing center which
serves several major international projects using a pool of shared computing
resources.

The compute and storage resources allocated to Rubin are progressively deployed
as need arises, typically matching the calendar year funding cycle.

Documentation specific to Rubin users is available at \href{https://doc.lsst.eu}{doc.lsst.eu}.

\subsubsection{Data release processing}

For Data Preview 0.2, which involves processing of the DESC DC2 simulated images,
the following resources are deployed, operational and routinely used:

\begin{itemize}
\item a \href{https://slurm.schedmd.com}{Slurm}-powered batch processing farm with
compute nodes equiped with CPUs of x86 architecture (64 bits). The allocation for
DP0.2 is equivalent to 3600 reference CPU cores (Intel Xeon E5-2680v3 @ 2.5GHz, see 
\citeds{DMTN-135}),
\item a POSIX-compliant file storage system implemented by 
\href{https://docs.ceph.com/en/latest/cephfs/index.html}{CephFS} with 5 PB 
available for image data and processing products,
\item a webDAV-compliant object storage system implemented by 
\href{https://www.dcache.org}{dCache} with 2.5 PB available for image data and
processing products,
\item two dedicated instances of PostgreSQL RBDMS with flash storage for butler
registry databases, one devoted for user's private registries and another for data release
processing registries,
\item a set of 4 dedicated data transfer nodes, each with 10 Gbps network
interface for exchanging data with the other data facilities.
\end{itemize}

Specifically for DP0.2, at the time of this writing a subset of the DESC DC2 simulated
images is being processed locally and independently of the other data facilities. The purpose of
this is to verify that the facility's infrastructure is well configured and to run the LSST pipelines
at scale as well as to exercise the tools for comparing the products of the local processing are
compatible with those produced by the Interim Data Facility.

\subsubsection{Catalog database}

A local production instance of Qserv composed of 15 physical, well-configured database server nodes 
is in production, populated with several catalog databases, including the DP0.2 catalog produced
at the Interim Data Facility.

An integration instance deployed on the on-prem OpenStack cloud for experimenting with new
releases of Qserv and Kubernetes.

\subsubsection{Science platform}

A 5-node evaluation instance of the Rubin Science Platform reachable at 
\href{https://data-dev.lsst.eu}{data-dev.lsst.eu} is deployed and is being integrated with
CC-IN2P3's specific services (e.g. identity and access management, user storage services, etc.) and
with the local Qserv instance. The intended use of that instance is to serve local users.

The Kubernetes cluster used by this science platform instance is shared with the Qserv production
instance.

\subsubsection{Software distribution}

The Rubin FrDF operates the source of truth software repository which distributes the LSST Science
Pipelines to the other Rubin data facilities via CERN's CernVM file system
(see \href{https://sw.lsst.eu}{sw.lsst.eu}).

Stable and weekly releases of the LSST Science Pipelines are made available via this distribution
mechanism and transparently accessed by users at the USDF and UKDF as well as from their
personnal computers. The purpose of this mechanism is to ensure strict compatibility of the software
releases used for data release production at the 3 facilities.


\subsubsection{Fink alert broker}

The \href{https://fink-broker.org}{Fink} alert broker will be hosted in FrDF's on-prem cloud
infrastructure on top of OpenStack. For year 2022, 250 CPU cores and 250 TB of storage on CephFS were
allocated for deploying the initial production-level instance of Fink. This project is ongoing at the
time of writing.

 \subsection{UKDF}

The UKDF will provide an Offline Production Enclave capable of processing 25\% of telescope observations and contributing these to the annual Data Release cycle.

The UKDF will also host a full Independent Data Access Centre and a Community Alert Broker.

The infrastructure that will constitute the UKDF will be sourced from the UK IRIS programme (http://www.iris.ac.uk/). IRIS is also expected to provide infrastructure to peer {\em experiments}, including the Euclid space telescope, the Square Kilometre Array, and the next generation of LHC experiments.

IRIS provides a mix of on-premises infrastructures hosted at partner institutions that can, at a high-level, be categorised as:

\begin{description}

\item[Cloud] IaaS-style compute and storage resources, typically provided as OpenStack virtualised compute and Ceph clusters.

\item[HPC] Tier~1-scale high-performance computing resources and working storage, accessible via  batch processing.

\item[Grid] Very high-throughput computing and storage based on LHC {\em grid} technologies.
  
\end{description}

IRIS provider institutions are connected via the UK academic network (Janet), which typically provides 100~Gbps potential bandwidth, as well as connections into the European academic network, G\'eant. 

IRIS also operates cross-cutting services to enable full exploitation of the infrastructure, including authentication and authorisation services (based on  IAM), usage accounting, security, policy development, and a helpdesk function. Some effort is likely to be required to harmonise these into a three-DF resource.

\subsubsection{Resource Project}

The UKDF team maintain a resource sizing mode, based on DMTN-135 for DRP and IDAC, and augmented by resource estimates for the Community Alert Broker and User-generated Products. This model is used to inform IRIS of the resource requirements for the UK in-kind contributions to telescope operations.

IRIS operates an annual infrastructure expansion and refresh, which coincides with the UK financial year (April--March). Each year, during October--December, experiments such as LSST:UK advise the IRIS Resource Scrutiny and Allocation Panel (RSAP) of their requirements for the following year (that is, April--March), plus give advanced notice of expected longer-term requirements for the subsequent four years. These requirements inform IRIS procurement plans for the subsequent year, as well as longer term funding requirements, to enable the experiments to continue to operate.

LSST:UK expect to utilise all three forms of IRIS infrastructure, loosely organised as follows:

\begin{itemize}

\item Data Release Processing will primarily be completed on Grid infrastructure, hosted at one of several UK provider sites. At the time of writing, LSST:UK has active allocations of grid resources at Lancaster University and Rutherford Appleton Laboratories. Storage will be provisioned to hold the UK's share of raw images (from the telescope), topical data products, and working storage for intermediate products and pipeline scratch space.

\item The UK IDAC will primarily be hosted on cloud infrastructure, provisioned using Kubernetes or equivalent container-orchestration technology.

\item Lasair will also be hosted on cloud infrastructure, alongside the UK IDAC to enable use cases spanning both Lasair and the RSP.

\item Various User-generated Products are planned, to extend the science potential of the Data Release products. HPC resources will be employed for those User-generated Products that require significant computational resources to produce (for example, running part of the LSST Science Pipeline).

\end{itemize}

% MGB; Table needs updating.

\begin{longtable} {|l  |l  |r  |r |r  |}
  \caption{\label{tab:ukdfInfrastructure}}
  \\
  \hline
  {\bf Infrastructure Type} & {\bf Key Services} & {\bf FY22} & {\bf FY23} & {\bf FY24} \\
  \hline
  Compute (Millions of Core Hours) & DRP, RSP & 2 & 11 & 21 \\
  \hline
  Storage (Petabytes) & /work, Butler, Qserv & 1.5 & 9.0 & 16.0 \\
  \hline
\end{longtable}

\subsubsection{Offline Production}

The UKDF team will provision compute and storage resources on Grid infrastructure, exposed to the Rubin Rucio service instance for distributed-date management and the Rubin PanDA service for job management.

These services will be supported by a team of expert staff -- currently, at 2.0~FTE, though rising to 5.0~FTE by the beginning of Rubin Operations.

\subsubsection{Independent Data Access Center}

A prototype IDAC is operational, running on IRIS cloud resources at the University of Edinburgh. It is expected that this prototype IDAC will mature into the production UK IDAC in advance of telescope operations.

At the time of writing, the UK prototype IDAC is running an instance of the Rubin Science Platform, including a 10-node Qserv implementation that is hosting catalogues from DP0.1 and early runs of HSC-VISTA fused catalogues, as a preview of the planned in-kind contribution of LSST/near-IR User-generated Product.

\subsubsection{Lasair Alert Community Broker}

A preview of the Lasair alert community broker is running on IRIS cloud resources, at the University of Edinburgh, processing and serving alerts from the Zwicky Transient Facility.

 
\section{Planning}\label{sec:plan}

The bottom layer of the Data Facilities is the hardware on which the
platforms run. These require a scalable architecture with sufficient
storage and \gls{CPU} to support the Data Preview/Release timeline.  This
element requries a design for the architecture followed by an
acquisition and installation plan.

The middle layer includes the infrastructure to support deployment of
hardware and tools for data movement and workflow management.

 The top layer involves the applications: science platform, \gls{Qserv} and pipelines.

In response to the timeline \secref{sec:timeline} the plan is as follows.

\subsection {Hardware architecture and technology}
See also \citeds{DMTN-189} (scope) and \citeds{DMTN-135}  (sizing).

\subsection{Key initial services}

These initial services/resources are planned to support \gls{DP1}. Support
for \gls{DP2} and DR1 are largely by increments in hardware per the sizing model.

\begin{itemize}
  \item Hardware
  \begin{itemize}
    \item file systems: An architecture choice must be made for object
      store, likely between ceph and minIO. This may affect the
      hardware choice (\gls{JBOD} vs appliance). 3.5 PB of object store and
      1.5 \gls{PB} of POSIX disk are envisaged.
    \item \gls{CPU} allocation: The bulk of the \gls{CPU} is in batch (1000 cores)
      and staff \gls{RSP} instances (500 cores).
      \item \gls{Qserv}: depending on the ultimate location of \gls{Qserv}, we
        expect to do scale testing in the cloud and at \gls{NCSA} prior to
        deciding on an implementation.
  \end{itemize}

\item
  \item \gls{Science Platform}
  \begin{itemize}
      \item \gls{K8S} is the standard for deploying applications and
        resources. The \gls{Science Platform} is built on top of
        it. Additionally, \gls{CI} activities are run via \gls{K8S}.
      \item \gls{RSP} has been installed in multiple locations and
        architectures. For \gls{DP1}, we expect science users to go to the
        IDF for data access, while the \gls{USDF} provides staff access.
  \end{itemize}

\item Workflow and Data Movement Tools
\begin{itemize}
      \item \gls{PanDA} is under serious consideration as the toolkit
        for at-scale workflow. It will get its first load testing in
        \gls{DP0}.2. It is expected that there will be worked needed to
        customize \gls{PanDA} to Rubin's situation. We also anticipate a
        layer on top of \gls{PanDA} to orchestrate campaign management.
	\item \gls{Rucio} is under consideration for data movement. It
       works with policies to schedule data movement and integrates
       with a transport layer (most commonly \gls{FTS}).
  \end{itemize}

\end{itemize}

\subsection {Enclave deployment}

\subsubsection{Prompt}

\subsubsection{Archive}

\subsubsection{US Data Access}

\subsubsection{Developer and Integration}

\subsubsection{Offline Production}

\section{Motivation}

The Legacy Survey of Space and Time (LSST) will be delivered to the science community as a collection of products, including a (typically) annual update to the survey, called a Data Release~(DR) and a pseudo-realtime alert stream. This document only considers the preparation and publication of Data Releases, as this work is to be shared between the three Data Facilities: in contrast, preparation and dissemination of the alert stream is the responsibility of the US Data Facility, and is discussed elsewhere.

Each Data Release is a substantial, multi-Petabyte resource consisting of various science-ready datasets:

\begin{description}

\item[Processed Visit Image] Processed versions of telescope images, corrected for instrumental signatures, etc.

\item[Deep Coadds] Stacked images of the same region of the sky, to create an image of increased sensitivity and detail.

\item[science catalogues] Collections of metadata that document detected astronomical objects along with standard measurements on those objects (location, colour, flux, etc.).

\item[Ancillary and intermediate products] Additional outputs that support different science use cases with different requirements, and enable downstream processing and the generation of derived datasets.

\end{description}

The preparation of each DR --- called Data Release Processing (DRP) --- is a substantial piece of work involving significant computational and storage infrastructure. It is a multi-stage process in which the LSST Science Pipeline (sometimes called, the LSST Stack) progresses through a {\em campaign} of image-manipulation and analysis processes. There are science-relevant decisions needed for many of these processes, meaning there may be a need to vary the configuration (branches in the pipeline) used at specific points, to support different science use cases. Thus, in addition to the end products noted above, it is expected that (at least) some of the intermediate products will be preserved (or should be readily reproducible).

DRP progress is recorded in a database registry called the Data Butler. This is intended to track the locations and contents of datasets as they are processed. It is heavily used by the LSST Stack and, on the completion of a DR, becomes the primary metadata enabling science users to interact with the survey images and coadds.

The Operations Plan (** ref **) details how the resources required for each DRP will be contributed by three facilities:

The overall responsibility for DRP is with the Rubin Observatory. They are responsible for developing the pipeline software, selecting appropriate middleware on which to deploy the processing workflow, and for undertaking quality assurance on the output data products. The UK (and French) Data Facilities are responsible for completing their agreed share of the workload, as follows:

\begin{itemize}

\item The US DF will complete 25\% of the processing

\item The French DF will complete 50\% of the processing

\item The UK DF will complete 25\% of the processing

\end{itemize}

At a high-level, DRP involves the following workflow:

\begin{itemize}

  \item Raw images, captured at the telescope, are transferred to the USDF over a dedicated network link, with very low latency.

  \item The portion of the raw images to be processed in France and the UK are then transmitted (on a timescale to be determined) on to those facilities, over the public Internet, along with calibration images and other ancillary products.

  \item Once each facility has the raw image data it is to process, it may proceed with processing (possibly after some form of data validation, or similar). Processing can effectively be completed independently at each facility, though there will be a requirement for the USDF to have an overview of fine-grain progress at each facility, plus there should be a capability to reassign processing work on the granularity of days/ weeks, in response to processing problems.

  \item Once processed, output data is transferred back to the USDF (again, over the public Internet) to be assembled into a Data Release.

  \item Processed data needs to be validated (by Rubin Observatory) before it is confirmed to be ready for publication. It is still to be determined, at the time of writing, if validation work will be undertaken when data is returned to the USDF or can be done earlier, at each data facility.
    
\end{itemize}

A Data Facility cannot work completely independently on their raw data and initial calibration data alone. Some processing steps aggregate data from across the DR. Because these steps are not necessarily at the end of the processing, and for other reasons, some data products (which may be part of the final DR or may be intermediates) will need to be distributed between Data Facilities during a campaign.

\begin{itemize}

  \item It is also intended that a full copy of all DR-related output data will be held at the French DF. This may serve as an online replica, for disaster recovery, though that may/ may not be the primary motivation for doing this (*** Fabio may wish to comment/ delete this ***).

  \item The UK DF requires a full copy of the output data from a DRP campaign, in order that it can serve science-ready data to a subset of the Rubin science community from a UK-based Data Access Centre. It is not anticipated, however, that the UK DF will need or want all the raw images: just those images that are to be processed in the UK.
    
\end{itemize}

It is intended that each DRP campaign will reprocess all images to date (that is, going back to the beginning of the survey). This is required to ensure consistent processing of all images (the processing stack will evolve throughout the survey), though does mean that the volume of DRP-related computing and storage will grow year on year.

The preparation of a Data Release – that is, a DRP campaign – should be completed within twelve months of the end of observing period (six month for DR1, which will be based on the first six months of observations). This includes assembly of complete datasets in both the US DF and French DF. It may not include the distribution of products to Independent Data Access Centres, though that should also be done in a timely manner, so as not to delay the release of a DR to the science community.


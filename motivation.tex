\section{Motivation}
\label{sec:motivation}

The Legacy Survey of Space and Time (LSST) will be delivered to the science community as a collection of products, including a (typically) annual update to the survey, called a Data Release~(DR) and a pseudo-realtime alert stream. Some or all of the three Data Facilities will play a role in the delivery of both of these products. To this end, this document considers:

\begin{itemize}

\item the preparation and publication of Data Releases (DRP)

\item the reception of the data from the telescope,

\item the preparation and dissemination of the alert stream

\end{itemize}

All three data facilities are involved in the first of these (DRP),
though only the USDF is involved in the other elements.


\subsection{Evolution from Construction to Commissioning to Operations}

The Construction of Rubin is winding down and Commissioning of systems is underway. Support of Commissioning will be a function of the USDF: transfer of image, calibration and timeseries data from the summit, as well as support of the Rubin staff and commissioners in the processing of the data. Along with the commissioning of the telescope, this process will also act as a pathfinder for activities needed in operations.


\subsection{Data Reception and Alert Processing}

There is a tight time budget for delivering data to the USDF and generating alerts: the goal is to issue alerts within one minute of closing the shutter. Parallel transfers will be required for each visit, broken down by CCD, and will be encrypted. A prompt processing pipeline will be executed upon receipt of the data, resulting in the initial alert stream. Some seven brokers are anticipated to be receiving the stream, in addition to sending data to the Minor Planet Center.


\subsection{Data Release Processing}

Each Data Release is a substantial, multi-Petabyte resource consisting of various science-ready datasets:

\begin{description}

\item[Processed Visit Image] Processed versions of telescope images,
  corrected for instrumental signatures, etc.

\item[Deep Coadds] Stacked images of the same region of the sky, to
  create an image of increased sensitivity and detail.

\item[science catalogues] Collections of metadata that document
  detected astronomical objects along with standard measurements on
  those objects (location, colour, flux, etc.).

\item[Ancillary and intermediate products] Additional outputs that
  support different science use cases with different requirements, and
  enable downstream processing and the generation of derived datasets.

\end{description}

The preparation of each DR is a substantial piece of work involving
significant computational and storage infrastructure. It is a
multi-stage process in which the LSST Science Pipeline (sometimes
called, the LSST Stack) progresses through a {\em campaign} of
image-manipulation and analysis processes. There are science-relevant
decisions needed for many of these processes, meaning there may be a
need to vary the configuration (branches in the pipeline) used at
specific points, to support different science use cases. Thus, in
addition to the end products noted above, it is expected that (at
least) some of the intermediate products will be preserved (or should
be readily reproducible).

DRP progress is recorded in a database registry called the Data
Butler. This is intended to track the locations, provenance and
contents of datasets as they are processed. It is heavily used by the
LSST Stack and, on the completion of a DR, becomes the primary
metadata enabling science users to interact with the survey images and
coadds.

The Operations Plan (** ref **) details how the resources required for
each DRP will be contributed by three facilities. The overall
responsibility for DRP is with the Rubin Observatory. They are
responsible for developing the pipeline software, selecting
appropriate middleware on which to deploy the processing workflow, and
for undertaking quality assurance on the output data products. The UK
(and French) Data Facilities are responsible for completing their
agreed share of the workload, as follows:

\begin{itemize}

\item The US DF will complete 25\% of the processing

\item The French DF will complete 50\% of the processing

\item The UK DF will complete 25\% of the processing

\end{itemize}

At a high-level, DRP involves the following workflow:

\begin{itemize}

  \item Raw images, captured at the telescope, are transferred to the
    USDF over a dedicated network link, with very low latency.

  \item The portion of the raw images to be processed in France and
    the UK are then transmitted (on a timescale to be determined) on
    to those facilities, over the public Internet, along with
    calibration images and other ancillary products.

  \item Once each facility has the raw image data it is to process, it
    may proceed with processing (possibly after some form of data
    validation, or similar). Processing can effectively be completed
    independently at each facility, though at several stages, there
    will be a need to exchange intermediate products between Data
    Facilities. Further, there will be a requirement for the USDF to
    have an overview of fine-grain progress at each facility. Finally,
    there should be a capability to reassign processing work on the
    granularity of days/ weeks, in response to processing problems.

  \item Once processed, output data is transferred back to the USDF
    (again, over the public Internet) to be assembled into a Data
    Release.

  \item Processed data needs to be validated (by Rubin Observatory
    staff) before it is confirmed to be ready for publication. It is
    still to be determined, at the time of writing, if validation work
    will be undertaken when data is returned to the USDF or can be
    done earlier, at each data facility.
    
\end{itemize}

A Data Facility cannot work completely independently on their raw data and initial calibration data alone. Some processing steps aggregate data from across the DR. Because these steps are not necessarily at the end of the processing, and for other reasons, some data products (which may be part of the final DR or may be intermediates) will need to be distributed between Data Facilities during a campaign.

\begin{itemize}

  \item It is also intended that a full copy of all DR-related output data will be held at the French DF. This may serve as an online replica, for disaster recovery, though that may/ may not be the primary motivation for doing this.

  \item The UK DF requires a full copy of the output data from a DRP campaign, so it can serve science-ready data to a subset of the Rubin science community from a UK-based Independent Data Access Center. It is not anticipated, however, that the UK DF will need or want all the raw images: just those images that are to be processed in the UK.
    
\end{itemize}

It is intended that each DRP campaign will reprocess all images to
date (that is, going back to the beginning of the survey). This is
required to ensure consistent processing of all images (the LSST Stack
will is likely to change as the survey progresses), and means that the
volume of DRP-related computing, storage and data transfers will grow
year on year.

The preparation of a Data Release – that is, a DRP campaign – should be completed within twelve months of the end of observing period (six month for DR1, which will be based on the first six months of observations). This includes assembly of complete datasets in both the US DF and French DF. It may not include the distribution of products to Independent Data Access Centers, though that should also be done in a timely manner, so as not to delay the release of a DR to the science community.

\subsection{Timeline}

The timeline for setting up the three-DF infrastructure and operation
is driven by the requirements of Commissioning and early operations,
which are documented in {\bf RDO-011}. That document describes the
production of a number of pre-operational and early-operations data
products, including three Data Previews (denoted DP0.2, DP1, and DP2)
and data releases (denoted by DR1, DR2, and DR3).

The Data Previews (and early Data Releases) serve a number of purposes
-- for example, testing the LSST Stack and informing science users of
LSST capabilities. They are also a convenient framework on which
to build three-DF capabilities.

At the time of writing, RDO-011 was last updated in April
2022. However, a newer timeline was presented at the Project and
Community Workshop in August 2022, and RDO-011 will be updated to
reflect this update in the near future:

\begin{itemize}

\item {\bf AuxTel/LATISS (in operation since 2021)} -- single CCD
  camera run in imagine or spectroscopic mode for measuring
  atmospheric effects. It has been used as a pathfinder for several
  systems.
  
\item {\bf DP0.2 (released June 2022)} -- a DRP experiment to
  reprocess data from DESC DC2 [REF] using the current LSST Pipeline
  software. This DRP experiment only involves a single data facility
  (the Interim Data Facility on Google Cloud), but is a vital source
  of information for the three-site DRP capability, as it exercises
  the fundamental elements of DRP.
  
\item {\bf 3DF Rehearsal - Spring 2023} -- a ramped re-run of the DRP workflow used to create DP0.2, though
  completed across the three Data Facilities, rather than on the IDF. It
  will be worthwhile to compare the properties of the DP0.2 and rehearsal
  campaigns, considering efficiency, resilience, and effectiveness,
  for example.

\item{\bf ComCam commissioning - engineering first light (July 2023)}
  Nine-CCD camera used as a pathfinder for the full
  system. Commissioning of the device will start 6 months earlier (Jan
  2023).
  
\item {\bf DP1 (April--June 2024)} -- data from the Commissioning
  Camera (ComCam), installed at the Observatory, will be processed
  twice: first, at SLAC (???) (Construction DF); and, then, across the
  three Data Facilities.

 \item{\bf LSSTCam commissioning - engineering first light (Mar 2024)}
   Commissioning of the full system. Commissioning of the device will
   start 6 months earlier (Sep 2023).
  
\item {\bf DP2 (December 2024--March 2025)} – Commissioning Data from
  the full LSST Camera, installed at the Observatory, will be
  processed across the three Data Facilities, aiming to mimic the
  conditions and timeline of a production DRP campaign as closely as
  possible.

\item {\bf DR1 (October 2025--January 2026)} -- The first production release,
  containing data from the first six months of observing.
  
\item {\bf DR2 (July--October 2026)} -- The second production release,
  containing data from the first twelve months of observing.
  
\item {\bf DR3 (July--October 2027)} -- The second production release,
  containing data from the first twelve months of observing.
  
\end{itemize}

This timeline is used for planning throughout this document.

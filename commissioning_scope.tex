\label{sec:commissioning_scope}
During the commissioning phase, the USDF will be the primary resource for personnel to interact with data coming off the summit. 
This is to minimize connections to the summit and ensure that summit computing resources are reserved for direct support of the summit team.
As part of regular commissioning and operations observations, a ``Rapid analysis" of the data will be performed in Chile.
This is a pared down version of the Single Frame Measurement Pipeline that is configured to run quickly at the expense of skipping and/or reducing the quality of the reduction. 
The data products from this reduction are stored in a database and used for helping observers address issues that may arise during the night and need to be immediately addressed. 
However, this data reduction is not sufficient for science analysis, and therefore within 12-24 hours of observation, the full data reduction will be performed using the production-level single frame measurement pipeTask. 

Another application where the USDF will assist in summit operations is in the analysis of the data acquired by pistoning the camera in and out of focus.
This procedure essentially uses the entire camera as a curvature wavefront sensor, then the data is analyzed using the Active Optics System analysis code (another pipeTask).
This dataset can be reduced on the summit but due to the superior computing resources it will be faster to perform the analysis at the USDF.
The triggering mechanism is to be determined, but is expected to utilize the methods developed by the prompt processing team.

The data products from all reductions will be stored in a database that allows their results to be compared, often referred to as the consolidated database.
As with many of the operational routines, the systems will be first implemented using the Auxiliary Telescope, then ComCam, and ultimately LSSTCam.

\subsubsection{AuxTel}

\begin{itemize}

\item start date: Jan 2021
\item end date: xx 202?
\item Target: use as pathfinder for processing, handling of data
  products and analysis prior to ComCam on-sky operation.
\item Functionality
  \begin{itemize}
  \item automatically receive data at the USDF
    \item infrastructure in place to analyse the data: DM stack,
      functioning butler and repo, RSP
    \item live EFD feed
      \item workflow tools to process segments of data (eg an entire
        night) or specific datasets
        \item modest compute and disk storage
        \end{itemize}
      \end{itemize}

\subsubsection{ComCam}

  \begin{itemize}
\item start date: Jan 2023
\item end date: Jul 2023
\item Target: prepare for processing, handling of data
  products and analysis prior to ComCam on-sky operation.
\item Functionality
  \begin{itemize}
  \item automatically receive data at the USDF
    \item infrastructure in place to analyse the data: DM stack,
      functioning butler and repo, RSP
      \item Portal function of RSP, with TAP service backed by a Qserv instance
    \item live EFD feed
      \item workflow tools to process segments of data (eg an entire
        night) or specific datasets
      \item 1000 compute cores and 1 PB disk storage
      \end{itemize}
      \end{itemize}

      \subsubsection{LSSTCam}

        \begin{itemize}
\item start date: Sep 2023
\item end date: Mar 2024
\item Target: prepare for processing, handling of data
  products and analysis prior to LSSTCam on-sky operation.
\item Functionality
  \begin{itemize}
  \item automatically receive data at the USDF
    \item infrastructure in place to analyse the data: DM stack,
      functioning butler and repo, RSP
      \item Portal function of RSP, with TAP service backed by a Qserv instance
    \item live EFD feed
      \item workflow tools to process segments of data (eg an entire
        night) or specific datasets
      \item 5000 compute cores and101 PB disk storage
              \end{itemize}
      \end{itemize}
